\documentclass[a4paper, 11pt]{article}
\usepackage[top=3cm, bottom=3cm, left = 2cm, right = 2cm]{geometry} 
\geometry{a4paper} 
\usepackage[utf8]{inputenc}
\usepackage{textcomp}
\usepackage{graphicx} 
\usepackage{amsmath,amssymb}  
\usepackage{bm}  
\usepackage{lipsum}
\usepackage[pdftex,bookmarks,colorlinks,breaklinks]{hyperref}  
%\hypersetup{linkcolor=black,citecolor=black,filecolor=black,urlcolor=black} % black links, for printed output
\usepackage{memhfixc} 
\usepackage{pdfsync}   
\usepackage{fancyhdr}
\usepackage{hyperref}
\pagestyle{fancy}

\title{Miner Report}
\author{Phan Dinh Minh Hieu}
%\date{}

\begin{document}
\maketitle
\tableofcontents
\pagebreak

\section{Introduction}

This document is about meta-node blockchain's miner, which can run on computer or mobile device 
to verify transactions and execute smart contracts.
Miner will receive amount of MTD native token for contributing. \linebreak
There are 2 types of miner: Verify and Execute
\pagebreak

\section{Join node}
In order to start mining, Miner have to do following step:
\begin{itemize}
    \item Install \href{http://https://metanode.co/}{Meta-node browser}.
    \item Install \href{http://https://metanode.co/}{Mining dapp}.
    \item Stake to node as Verify and/or Execute Miner.
    \item Connect to the node and start mining with the staked account.
  \end{itemize}
\pagebreak
\section{Mining reward and halving}
\subsection{Basics of MetaNode Mining}
Mining is is the process by which people use mobiles, computers or mining hardware 
to participate in MetaNode's blockchain network as a transaction validator, processor or forwarder.


Despite using Proof Of Stake, MTDs have to be generated by the mining process. Each block generated will reward some MTDs to the leader who created it.

From start, there is a limit cap for total MTDs are 100 000 000 000 MTD. And after 100 years, there's no more new MTD will be generated.
\subsection{Halving}
After the network mines 315 360 000 blocks roughly every once year, the block reward given to the leader for creating a block is cut in half. This event is called halving because it cuts the rate at which new MTDs are released into circulation in half.

This rewards system will continue until about 2123 when the proposed limit of 100 billion coins is reached. At that point, leaders will be rewarded with fees for processing transactions, which network users will pay. These fees ensure validators, nodes and miners are still incentivized to participate and keep the network going.

The halving event is significant because it marks another drop in the rate of new MTD being produced as it approaches its finite supply. 

Example of halving:
  In the first year, for each block generated the leader will be rewarded with 158 MTDs, but in the second year reward will be 79 MTDs left.
  And once the total MTD is equal 100,000,000,000, the leader will receive only the transaction fee in the block it generated.
\subsection{Formular}

$$ totalMTD = \sum_{i=1}^{100} rewardFromStart * totalBlockPerYear * \frac{1}{2^i} $$
$$ 100,000,000,000 = \sum_{i=1}^{100} rewardFromStart * 315,360,000 * \frac{1}{2^i} $$

After solving this we get $rewardFromStart \thickapprox 158$ MTDs.

\pagebreak

\section{Reward distribute for node and miner}
Whenever the miner success validates then sign or executes smart-contract, which creates an accepted vote\footnote{Usually need 2/3 miner with the same vote to make that vote be valid}, the node will add contribution point(CP) for the miner. 
\\
When node receive reward transaction from parent, it will save n percentage reward for itself and calculate CP percentage of each miner address 
then distribute leftover amount match percentage for miners by add it to pending reward.
\\
Node will create transaction to send reward for any miner that have pending reward greater than threshhold and transaction fee. 
Transaction fee for send reward will be direct subtract to miner receive amount.
\\
Reward transaction will be add to pending pack pool. If transaction fail, reward balance will be added back to pending pool.
\\
Example:
\begin{itemize}
    \item Node A have 2 miner, miner1 and miner2.
    \item Threshhold is 5 and transaction's fee is 1\footnote{In reality transaction's fee is calculate by gas used and gas fee}.
    \item miner1 CP is 6 and miner2 cp is 4.
    \item miner1 pending reward is 4, miner 2 pending reward is 0.
    \item Node A receive 10 reward transaction.
    \item Node A have reward distribute percentage is 80.
\end{itemize}
Then formula to calculate reward of miner1 and miner2 is:
$$miner1 = (5/(5 + 5)) * (10 * 80/100) = 4$$
$$miner2 = (5/(5 + 5)) * (10 * 80/100) = 4$$
Pending balance after distribute:
$$miner1 = 4 + 4 = 8$$
$$miner2 = 0 + 4 = 4$$
Because $8 > threshhold(5)$ so miner1 will receive reward transaction with amount = $$8 - transaction fee(1) = 7$$
\pagebreak

\section{Slashing distribute}
Whenever miner's vote is invalid\footnote{invalid vote is all vote different than valid vote} depredation point(DP) will be added instead CP.
\\
When node receive confirm block, it check all address with DP, any address have greater DP threshhold will receive a slash transaction which subtract stake amount\footnote{threshhold and slashing amount is configurable by each node}.


\pagebreak


\section{Verify Miner}

\subsection{Introduction}
Verify miner will receive these command for verify:
    \begin{itemize}
        \item VerifyTransactionSign: 
        \item VerifyPackSign: require miner to verify aggregate of single pack; data contents aggregate sign, list pubkey, list transaction's hash
        \item VerifyPacksSign: require miner to verify aggregate of multiple pack; data contents list of aggregate sign, list pubkey, list transaction's hash
    \end{itemize}
\subsection{Flow chart}
Updating
\subsection{Reward tracking}
\begin{itemize}
    \item VerifyTransactionSign: $1CP$
    \item VerifyPackSign: $1CP$
    \item VerifyPacksSign: $1CP * totalPack$
\end{itemize}

\subsection{Hardware require}
Updating
\pagebreak

\section{Execute Miner}
\subsection{Introduction}
Execute miner will receive only  ExecuteTransactions which content multiple transaction for execute include deploy and call
\subsection{Flow chart}
Updating
\subsection{Reward tracking}
\begin{itemize}
    \item ExecuteTransactions: $1CP * total Transaction$
\end{itemize}

\subsection{Hardware require}
Updating

\pagebreak

\bibliographystyle{abbrv}
% \bibliography{references}  % need to put bibtex references in references.bib 
\end{document}